% Options for packages loaded elsewhere
\PassOptionsToPackage{unicode}{hyperref}
\PassOptionsToPackage{hyphens}{url}
%
\documentclass[
]{article}
\usepackage{amsmath,amssymb}
\usepackage{iftex}
\ifPDFTeX
  \usepackage[T1]{fontenc}
  \usepackage[utf8]{inputenc}
  \usepackage{textcomp} % provide euro and other symbols
\else % if luatex or xetex
  \usepackage{unicode-math} % this also loads fontspec
  \defaultfontfeatures{Scale=MatchLowercase}
  \defaultfontfeatures[\rmfamily]{Ligatures=TeX,Scale=1}
\fi
\usepackage{lmodern}
\ifPDFTeX\else
  % xetex/luatex font selection
\fi
% Use upquote if available, for straight quotes in verbatim environments
\IfFileExists{upquote.sty}{\usepackage{upquote}}{}
\IfFileExists{microtype.sty}{% use microtype if available
  \usepackage[]{microtype}
  \UseMicrotypeSet[protrusion]{basicmath} % disable protrusion for tt fonts
}{}
\makeatletter
\@ifundefined{KOMAClassName}{% if non-KOMA class
  \IfFileExists{parskip.sty}{%
    \usepackage{parskip}
  }{% else
    \setlength{\parindent}{0pt}
    \setlength{\parskip}{6pt plus 2pt minus 1pt}}
}{% if KOMA class
  \KOMAoptions{parskip=half}}
\makeatother
\usepackage{xcolor}
\usepackage[margin=1in]{geometry}
\usepackage{color}
\usepackage{fancyvrb}
\newcommand{\VerbBar}{|}
\newcommand{\VERB}{\Verb[commandchars=\\\{\}]}
\DefineVerbatimEnvironment{Highlighting}{Verbatim}{commandchars=\\\{\}}
% Add ',fontsize=\small' for more characters per line
\usepackage{framed}
\definecolor{shadecolor}{RGB}{248,248,248}
\newenvironment{Shaded}{\begin{snugshade}}{\end{snugshade}}
\newcommand{\AlertTok}[1]{\textcolor[rgb]{0.94,0.16,0.16}{#1}}
\newcommand{\AnnotationTok}[1]{\textcolor[rgb]{0.56,0.35,0.01}{\textbf{\textit{#1}}}}
\newcommand{\AttributeTok}[1]{\textcolor[rgb]{0.13,0.29,0.53}{#1}}
\newcommand{\BaseNTok}[1]{\textcolor[rgb]{0.00,0.00,0.81}{#1}}
\newcommand{\BuiltInTok}[1]{#1}
\newcommand{\CharTok}[1]{\textcolor[rgb]{0.31,0.60,0.02}{#1}}
\newcommand{\CommentTok}[1]{\textcolor[rgb]{0.56,0.35,0.01}{\textit{#1}}}
\newcommand{\CommentVarTok}[1]{\textcolor[rgb]{0.56,0.35,0.01}{\textbf{\textit{#1}}}}
\newcommand{\ConstantTok}[1]{\textcolor[rgb]{0.56,0.35,0.01}{#1}}
\newcommand{\ControlFlowTok}[1]{\textcolor[rgb]{0.13,0.29,0.53}{\textbf{#1}}}
\newcommand{\DataTypeTok}[1]{\textcolor[rgb]{0.13,0.29,0.53}{#1}}
\newcommand{\DecValTok}[1]{\textcolor[rgb]{0.00,0.00,0.81}{#1}}
\newcommand{\DocumentationTok}[1]{\textcolor[rgb]{0.56,0.35,0.01}{\textbf{\textit{#1}}}}
\newcommand{\ErrorTok}[1]{\textcolor[rgb]{0.64,0.00,0.00}{\textbf{#1}}}
\newcommand{\ExtensionTok}[1]{#1}
\newcommand{\FloatTok}[1]{\textcolor[rgb]{0.00,0.00,0.81}{#1}}
\newcommand{\FunctionTok}[1]{\textcolor[rgb]{0.13,0.29,0.53}{\textbf{#1}}}
\newcommand{\ImportTok}[1]{#1}
\newcommand{\InformationTok}[1]{\textcolor[rgb]{0.56,0.35,0.01}{\textbf{\textit{#1}}}}
\newcommand{\KeywordTok}[1]{\textcolor[rgb]{0.13,0.29,0.53}{\textbf{#1}}}
\newcommand{\NormalTok}[1]{#1}
\newcommand{\OperatorTok}[1]{\textcolor[rgb]{0.81,0.36,0.00}{\textbf{#1}}}
\newcommand{\OtherTok}[1]{\textcolor[rgb]{0.56,0.35,0.01}{#1}}
\newcommand{\PreprocessorTok}[1]{\textcolor[rgb]{0.56,0.35,0.01}{\textit{#1}}}
\newcommand{\RegionMarkerTok}[1]{#1}
\newcommand{\SpecialCharTok}[1]{\textcolor[rgb]{0.81,0.36,0.00}{\textbf{#1}}}
\newcommand{\SpecialStringTok}[1]{\textcolor[rgb]{0.31,0.60,0.02}{#1}}
\newcommand{\StringTok}[1]{\textcolor[rgb]{0.31,0.60,0.02}{#1}}
\newcommand{\VariableTok}[1]{\textcolor[rgb]{0.00,0.00,0.00}{#1}}
\newcommand{\VerbatimStringTok}[1]{\textcolor[rgb]{0.31,0.60,0.02}{#1}}
\newcommand{\WarningTok}[1]{\textcolor[rgb]{0.56,0.35,0.01}{\textbf{\textit{#1}}}}
\usepackage{graphicx}
\makeatletter
\def\maxwidth{\ifdim\Gin@nat@width>\linewidth\linewidth\else\Gin@nat@width\fi}
\def\maxheight{\ifdim\Gin@nat@height>\textheight\textheight\else\Gin@nat@height\fi}
\makeatother
% Scale images if necessary, so that they will not overflow the page
% margins by default, and it is still possible to overwrite the defaults
% using explicit options in \includegraphics[width, height, ...]{}
\setkeys{Gin}{width=\maxwidth,height=\maxheight,keepaspectratio}
% Set default figure placement to htbp
\makeatletter
\def\fps@figure{htbp}
\makeatother
\setlength{\emergencystretch}{3em} % prevent overfull lines
\providecommand{\tightlist}{%
  \setlength{\itemsep}{0pt}\setlength{\parskip}{0pt}}
\setcounter{secnumdepth}{-\maxdimen} % remove section numbering
\ifLuaTeX
  \usepackage{selnolig}  % disable illegal ligatures
\fi
\usepackage{bookmark}
\IfFileExists{xurl.sty}{\usepackage{xurl}}{} % add URL line breaks if available
\urlstyle{same}
\hypersetup{
  pdftitle={Testing for scale-dependent phylogenetic signal in intetegular distance in bee communities in Montana, USA},
  hidelinks,
  pdfcreator={LaTeX via pandoc}}

\title{Testing for scale-dependent phylogenetic signal in intetegular
distance in bee communities in Montana, USA}
\author{}
\date{\vspace{-2.5em}2024-10-17}

\begin{document}
\maketitle

\subsection{Introduction}\label{introduction}

Species interactions vary across space and time due to both species
turnover and to a phenomenon known as interaction rewiring. Interaction
rewiring may be driven by temporal or spatial changes in phenology,
abundance, or species traits (both intra- and interspecific). The
impacts of interspecific trait matching, phenology, and abundance on
ecological netowrks have been repeatedly demonstrated: changes in
flowering time or tongue morphology have clear impacts on
plant-pollinator interactions, for example. However, the role of
intraspecific trait differences in shaping interactions---and their
relative importance---remain poorly known.

Bees are a species-rich (n\textgreater20,000) clade of Hymenopterans
with a cosmopolitan distribution. Ecologically important pollinators,
intraspecific variation in body size is common in some taxa. Because
body size is highly correlated with tongue length and flying ability, it
is plausible it mediates foraging choices and thus may explain a portion
of plant-pollinator interaction rewiring across space.

\subsection{Methods}\label{methods}

\begin{itemize}
\tightlist
\item
  Bees and flowering plants were sampled when observed interacting
  across 4 years and 142 sites in Montana, USA. Sites were arranged into
  spatially nested design: 9 sites comprised a block and 8 blocks
  comprised a locality; there were two localities in the study.
\item
  Bees were associated with their foraging substrate, collected and
  pinned for identification. A proxy for body size, intertegular
  distance (ITD)---the distance between tegulae (wing bases) on
  the---was measured in each female individual. (As 75\% of collected
  individuals were female, male bees were excluded.)
\end{itemize}

\subsection{Questions}\label{questions}

Our study aims to answer the following questions:

\begin{enumerate}
\def\labelenumi{\arabic{enumi})}
\tightlist
\item
  How much phylogenetic signal is there in the magnitude of
  intraspecific variation in female bee body size?
\item
  At what spatial scale(s) does phylogenetic signal emerge?
\item
  Do bee species present at a greater number of sites have more
  intraspecific variation in female bee body size?
\item
  Is intraspecific variation in female bee body size predicted by
  interaction partners after controlling for phylogeny?
\item
  Does the strength of the relationship between intraspecific variation
  in female bee body size vary across scales?
\end{enumerate}

\subsection{Analysis}\label{analysis}

We'll begin by loading libraries for data manipulation, visualization,
and Bayesian phylogenetic mixed models:

\begin{Shaded}
\begin{Highlighting}[]
\FunctionTok{library}\NormalTok{(tidyverse)}
\FunctionTok{library}\NormalTok{(ggplot2)}
\FunctionTok{library}\NormalTok{(reshape2)}
\FunctionTok{library}\NormalTok{(brms)}
\FunctionTok{library}\NormalTok{(ape)}
\FunctionTok{library}\NormalTok{(cowplot)}
\FunctionTok{library}\NormalTok{(ggtree)}
\FunctionTok{library}\NormalTok{(phytools)}
\end{Highlighting}
\end{Shaded}

Let's also write a function to calculate Bao's estimator of variation
(\(CV_4\)):

\begin{Shaded}
\begin{Highlighting}[]
\NormalTok{bao }\OtherTok{\textless{}{-}} \ControlFlowTok{function}\NormalTok{(data)\{}
\NormalTok{  N}\OtherTok{=}\FunctionTok{length}\NormalTok{(data)}
\NormalTok{  y\_bar}\OtherTok{=}\FunctionTok{mean}\NormalTok{(data)}
\NormalTok{  s2\_hat}\OtherTok{=}\FunctionTok{var}\NormalTok{(data)}
\NormalTok{  cv\_2}\OtherTok{=}\NormalTok{s2\_hat}\SpecialCharTok{/}\NormalTok{y\_bar}\SpecialCharTok{\^{}}\DecValTok{2}
\NormalTok{  cv\_1}\OtherTok{=}\FunctionTok{sqrt}\NormalTok{(cv\_2)}
\NormalTok{  gamma\_1}\OtherTok{=}\FunctionTok{sum}\NormalTok{(((data}\SpecialCharTok{{-}}\NormalTok{y\_bar)}\SpecialCharTok{/}\NormalTok{s2\_hat}\SpecialCharTok{\^{}}\FloatTok{0.5}\NormalTok{)}\SpecialCharTok{\^{}}\DecValTok{3}\NormalTok{)}\SpecialCharTok{/}\NormalTok{N}
\NormalTok{  gamma\_2}\OtherTok{=}\FunctionTok{sum}\NormalTok{(((data}\SpecialCharTok{{-}}\NormalTok{y\_bar)}\SpecialCharTok{/}\NormalTok{s2\_hat}\SpecialCharTok{\^{}}\FloatTok{0.5}\NormalTok{)}\SpecialCharTok{\^{}}\DecValTok{4}\NormalTok{)}\SpecialCharTok{/}\NormalTok{N}
\NormalTok{  bias}\OtherTok{=}\NormalTok{cv\_2}\SpecialCharTok{\^{}}\NormalTok{(}\DecValTok{3}\SpecialCharTok{/}\DecValTok{2}\NormalTok{)}\SpecialCharTok{/}\NormalTok{N}\SpecialCharTok{*}\NormalTok{(}\DecValTok{3}\SpecialCharTok{*}\NormalTok{cv\_2}\SpecialCharTok{\^{}}\FloatTok{0.5}\DecValTok{{-}2}\SpecialCharTok{*}\NormalTok{gamma\_1)}
\NormalTok{  bias2}\OtherTok{=}\NormalTok{cv\_1}\SpecialCharTok{\^{}}\DecValTok{3}\SpecialCharTok{/}\NormalTok{N}\SpecialCharTok{{-}}\NormalTok{cv\_1}\SpecialCharTok{/}\DecValTok{4}\SpecialCharTok{/}\NormalTok{N}\SpecialCharTok{{-}}\NormalTok{cv\_1}\SpecialCharTok{\^{}}\DecValTok{2}\SpecialCharTok{*}\NormalTok{gamma\_1}\SpecialCharTok{/}\DecValTok{2}\SpecialCharTok{/}\NormalTok{N}\SpecialCharTok{{-}}\NormalTok{cv\_1}\SpecialCharTok{*}\NormalTok{gamma\_2}\SpecialCharTok{/}\DecValTok{8}\SpecialCharTok{/}\NormalTok{N}
\NormalTok{  cv4}\OtherTok{=}\NormalTok{cv\_1}\SpecialCharTok{{-}}\NormalTok{bias2}
  \FunctionTok{return}\NormalTok{(cv4)}
\NormalTok{\}}
\end{Highlighting}
\end{Shaded}

Now we load our data:

\begin{Shaded}
\begin{Highlighting}[]
\NormalTok{bee\_data }\OtherTok{\textless{}{-}} \FunctionTok{read\_csv}\NormalTok{(}\StringTok{"\textasciitilde{}/Dropbox/bee\_phylo\_itv/bee\_females\_for\_ethan.csv"}\NormalTok{) }\CommentTok{\# bee trait data}
\NormalTok{bee\_phylo }\OtherTok{\textless{}{-}} \FunctionTok{read.tree}\NormalTok{(}\StringTok{"\textasciitilde{}/Dropbox/bee\_phylo\_itv/BEE\_prunedtree.nwk"}\NormalTok{) }\CommentTok{\# pruned newick{-}format phlogeny of bee species present in MT}
\end{Highlighting}
\end{Shaded}

Let's look at the trait data first:

\begin{Shaded}
\begin{Highlighting}[]
\NormalTok{bee\_data}
\end{Highlighting}
\end{Shaded}

\begin{verbatim}
## # A tibble: 4,367 x 27
##    TRANSECT_COMBO LOCATION block_2 IT_microns IT_mm ORDER       FAMILY     GENUS
##    <chr>          <chr>    <chr>        <dbl> <dbl> <chr>       <chr>      <chr>
##  1 HEMX1_01       HE       E            1776.  1.78 Hymenoptera Halictidae Agap~
##  2 HEMX1_02       HE       E            1857.  1.86 Hymenoptera Halictidae Agap~
##  3 HEMX1_06       HE       E            1857.  1.86 Hymenoptera Halictidae Agap~
##  4 HENEWHI1_04    HE       A            1959.  1.96 Hymenoptera Halictidae Agap~
##  5 HENEWHI2_02    HE       B            1959.  1.96 Hymenoptera Halictidae Agap~
##  6 HENEWHI2_04    HE       B            1939.  1.94 Hymenoptera Halictidae Agap~
##  7 HEOLDHI1_07    HE       M            1959.  1.96 Hymenoptera Halictidae Agap~
##  8 HEOLDHI2_06    HE       N            1816.  1.82 Hymenoptera Halictidae Agap~
##  9 HEOLDHI2_08    HE       N            1898.  1.90 Hymenoptera Halictidae Agap~
## 10 HEMX1_03       HE       E            2020.  2.02 Hymenoptera Halictidae Agap~
## # i 4,357 more rows
## # i 19 more variables: GENUS_SPECIES <chr>, SEX <lgl>, NESTING <chr>,
## #   NestingAlt <chr>, SOCIALITY <chr>, SocialityAlt <chr>, FLOWER_CODE2 <chr>,
## #   USDA_CODE <chr>, FLOWER_SPECIES <chr>, COUNTRY <chr>, STATE <chr>,
## #   COUNTY <chr>, LATITUDE <dbl>, LONGITUDE <dbl>, ELEVATION_m <dbl>,
## #   JULIAN_DATE <dbl>, DATE <chr>, TIME <time>, YEAR <dbl>
\end{verbatim}

Important columns include \texttt{TRANSECT\_COMBO}, \texttt{LOCATION},
\texttt{block\_2}, \texttt{IT\_microns}, \texttt{GENUS\_SPECIES}, and
\texttt{FLOWER\_SPECIES}. We can summarize these data to get an idea of
their size and diversity. First, the number of bee species in the
dataset:

\begin{Shaded}
\begin{Highlighting}[]
\NormalTok{bee\_data}\SpecialCharTok{$}\NormalTok{GENUS\_SPECIES }\SpecialCharTok{\%\textgreater{}\%} \FunctionTok{unique}\NormalTok{() }\SpecialCharTok{\%\textgreater{}\%} \FunctionTok{length}\NormalTok{()}
\end{Highlighting}
\end{Shaded}

\begin{verbatim}
## [1] 149
\end{verbatim}

\ldots the number of plant species:

\begin{Shaded}
\begin{Highlighting}[]
\NormalTok{bee\_data}\SpecialCharTok{$}\NormalTok{FLOWER\_SPECIES }\SpecialCharTok{\%\textgreater{}\%} \FunctionTok{unique}\NormalTok{() }\SpecialCharTok{\%\textgreater{}\%} \FunctionTok{length}\NormalTok{()}
\end{Highlighting}
\end{Shaded}

\begin{verbatim}
## [1] 95
\end{verbatim}

\ldots and the total number of observations:

\begin{Shaded}
\begin{Highlighting}[]
\NormalTok{bee\_data }\SpecialCharTok{\%\textgreater{}\%} \FunctionTok{nrow}\NormalTok{()}
\end{Highlighting}
\end{Shaded}

\begin{verbatim}
## [1] 4367
\end{verbatim}

We may need to filter down these data to only include female bees. Let's
check to see what levels are present in the ``SEX'' column:

\begin{Shaded}
\begin{Highlighting}[]
\NormalTok{bee\_data}\SpecialCharTok{$}\NormalTok{SEX }\SpecialCharTok{\%\textgreater{}\%} \FunctionTok{unique}\NormalTok{()}
\end{Highlighting}
\end{Shaded}

\begin{verbatim}
## [1] FALSE
\end{verbatim}

None, apparently---we must be working with a pre-filtered CSV. In that
case, let's go ahead and calculate Bao's CV for all species with a
sample size of n \(\geq\) 20:

\begin{Shaded}
\begin{Highlighting}[]
\FunctionTok{set.seed}\NormalTok{(}\DecValTok{1}\NormalTok{) }\CommentTok{\# set seed for reproducibility}
\NormalTok{bao\_values }\OtherTok{\textless{}{-}}\NormalTok{ bee\_data }\SpecialCharTok{\%\textgreater{}\%}
  \FunctionTok{group\_by}\NormalTok{(GENUS\_SPECIES) }\SpecialCharTok{\%\textgreater{}\%}  
  \FunctionTok{mutate}\NormalTok{(}\AttributeTok{total\_observations =} \FunctionTok{n}\NormalTok{()) }\SpecialCharTok{\%\textgreater{}\%} 
  \FunctionTok{filter}\NormalTok{(}\FunctionTok{n}\NormalTok{() }\SpecialCharTok{\textgreater{}=} \DecValTok{20}\NormalTok{) }\SpecialCharTok{\%\textgreater{}\%}                        \CommentTok{\# Filter groups with 20 or more observations}
  \FunctionTok{sample\_n}\NormalTok{(}\DecValTok{20}\NormalTok{) }\SpecialCharTok{\%\textgreater{}\%}                             \CommentTok{\# Randomly sample 20 rows from the filtered data}
  \FunctionTok{mutate}\NormalTok{(}\AttributeTok{bao\_cv =} \FunctionTok{bao}\NormalTok{(IT\_microns), }
    \AttributeTok{num\_sites =} \FunctionTok{n\_distinct}\NormalTok{(TRANSECT\_COMBO)) }\SpecialCharTok{\%\textgreater{}\%}
  \FunctionTok{select}\NormalTok{(GENUS\_SPECIES, bao\_cv, num\_sites, total\_observations) }\SpecialCharTok{\%\textgreater{}\%}
  \FunctionTok{distinct}\NormalTok{()}
\NormalTok{bao\_values}
\end{Highlighting}
\end{Shaded}

\begin{verbatim}
## # A tibble: 59 x 4
## # Groups:   GENUS_SPECIES [59]
##    GENUS_SPECIES            bao_cv num_sites total_observations
##    <chr>                     <dbl>     <int>              <int>
##  1 Agapostemon_virescens    0.0364        16                 26
##  2 Andrena_lawrencei        0.0446        10                 82
##  3 Andrena_miranda          0.0444        16                 30
##  4 Andrena_topazana         0.0552        14                 20
##  5 Anthidium_utahense       0.0680        15                 38
##  6 Anthophora_terminalis    0.0260        14                 25
##  7 Apis_mellifera           0.0440        15                 62
##  8 Ashmeadiella_bucconis    0.0724        16                 50
##  9 Ashmeadiella_cactorum    0.0474        15                 36
## 10 Ashmeadiella_californica 0.0797        12                 20
## # i 49 more rows
\end{verbatim}

How are these data distributed?

\begin{Shaded}
\begin{Highlighting}[]
\NormalTok{p1 }\OtherTok{\textless{}{-}} \FunctionTok{ggplot}\NormalTok{(bao\_values, }\FunctionTok{aes}\NormalTok{(}\AttributeTok{x=}\FunctionTok{log}\NormalTok{(bao\_cv))) }\SpecialCharTok{+} 
  \FunctionTok{theme\_bw}\NormalTok{() }\SpecialCharTok{+}
  \FunctionTok{geom\_histogram}\NormalTok{(}\AttributeTok{bins=}\DecValTok{40}\NormalTok{)}
\NormalTok{p1}
\end{Highlighting}
\end{Shaded}

\includegraphics{01_analysis_files/figure-latex/unnamed-chunk-10-1.pdf}

Next, let's prepare our phylogenetic data. We need to create a consensus
phylogeny that retains branch lengths (by averaging), drop tips that
aren't present in the 59 species in our summary dataset, cand alculate
and visualize a correlation matrix given a Brownian motion model of
trait evolution (NOT a covariance matrix; see dicussion here:
\url{https://discourse.mc-stan.org/t/covariance-matrix-phylogenetic-models/20477/4}).

\begin{Shaded}
\begin{Highlighting}[]
\NormalTok{bee\_phylo\_consensus }\OtherTok{\textless{}{-}} \FunctionTok{consensus.edges}\NormalTok{(bee\_phylo) }
\NormalTok{bee\_phylo\_consensus }\OtherTok{\textless{}{-}} \FunctionTok{keep.tip}\NormalTok{(bee\_phylo\_consensus, }\FunctionTok{unique}\NormalTok{(bao\_values}\SpecialCharTok{$}\NormalTok{GENUS\_SPECIES))}
\NormalTok{A }\OtherTok{\textless{}{-}}\NormalTok{ ape}\SpecialCharTok{::}\FunctionTok{vcv.phylo}\NormalTok{(bee\_phylo\_consensus, }\AttributeTok{corr =} \ConstantTok{TRUE}\NormalTok{)}
\NormalTok{melted\_A }\OtherTok{\textless{}{-}} \FunctionTok{melt}\NormalTok{(A)}
\FunctionTok{head}\NormalTok{(melted\_A)}
\end{Highlighting}
\end{Shaded}

\begin{verbatim}
##                 Var1              Var2      value
## 1  Andrena_lawrencei Andrena_lawrencei 1.00000000
## 2    Andrena_miranda Andrena_lawrencei 0.77446907
## 3   Andrena_topazana Andrena_lawrencei 0.77443624
## 4 Colletes_kincaidii Andrena_lawrencei 0.02584236
## 5 Colletes_phaceliae Andrena_lawrencei 0.02584186
## 6  Colletes_fulgidus Andrena_lawrencei 0.02584443
\end{verbatim}

\begin{Shaded}
\begin{Highlighting}[]
\NormalTok{cov }\OtherTok{\textless{}{-}} \FunctionTok{ggplot}\NormalTok{(}\AttributeTok{data =}\NormalTok{ melted\_A, }\FunctionTok{aes}\NormalTok{(}\AttributeTok{x=}\NormalTok{Var1, }\AttributeTok{y=}\NormalTok{Var2, }\AttributeTok{fill=}\NormalTok{value)) }\SpecialCharTok{+} 
  \FunctionTok{geom\_tile}\NormalTok{() }\SpecialCharTok{+}
  \FunctionTok{theme}\NormalTok{(}\AttributeTok{axis.text.x =} \FunctionTok{element\_text}\NormalTok{(}\AttributeTok{angle =} \DecValTok{90}\NormalTok{, }\AttributeTok{vjust =} \FloatTok{0.5}\NormalTok{, }\AttributeTok{hjust=}\DecValTok{1}\NormalTok{), }\AttributeTok{axis.title =} \FunctionTok{element\_blank}\NormalTok{())}
\NormalTok{cov}
\end{Highlighting}
\end{Shaded}

\includegraphics{01_analysis_files/figure-latex/unnamed-chunk-11-1.pdf}

Here, areas of light blue show high correlation values (e.g., are
pairwsie comparisons of members of the same clade).

\subsection{1) How much phylogenetic signal is there in the magnitude of
intraspecific variation in female bee body
size?}\label{how-much-phylogenetic-signal-is-there-in-the-magnitude-of-intraspecific-variation-in-female-bee-body-size}

To test for phylogenetic signal in our ITV, we'll build a simple
Bayesian linear model to predict Bao's CV from the number of sites where
a given species is found. In out subsampled data, there \emph{may} be a
weak association between these variables, as well as the total number of
observations of a species (before subsetting) and its Bao CV value
(calculated after subsetting):

\begin{Shaded}
\begin{Highlighting}[]
\NormalTok{p2 }\OtherTok{\textless{}{-}} \FunctionTok{ggplot}\NormalTok{(bao\_values, }\FunctionTok{aes}\NormalTok{(}\AttributeTok{x=}\NormalTok{num\_sites, }\AttributeTok{y=}\FunctionTok{log}\NormalTok{(bao\_cv), )) }\SpecialCharTok{+} 
  \FunctionTok{theme\_bw}\NormalTok{() }\SpecialCharTok{+}
  \FunctionTok{geom\_point}\NormalTok{(}\AttributeTok{pch=}\DecValTok{21}\NormalTok{)}
\NormalTok{p3 }\OtherTok{\textless{}{-}} \FunctionTok{ggplot}\NormalTok{(bao\_values, }\FunctionTok{aes}\NormalTok{(}\AttributeTok{x=}\NormalTok{total\_observations, }\AttributeTok{y=}\FunctionTok{log}\NormalTok{(bao\_cv), )) }\SpecialCharTok{+} 
  \FunctionTok{theme\_bw}\NormalTok{() }\SpecialCharTok{+}
  \FunctionTok{geom\_point}\NormalTok{(}\AttributeTok{pch=}\DecValTok{21}\NormalTok{) }\SpecialCharTok{+}
  \FunctionTok{xlim}\NormalTok{(}\DecValTok{0}\NormalTok{,}\DecValTok{100}\NormalTok{) }\CommentTok{\# note trimmed axis to ignore outliers}

\FunctionTok{plot\_grid}\NormalTok{(p2, p3)}
\end{Highlighting}
\end{Shaded}

\includegraphics{01_analysis_files/figure-latex/unnamed-chunk-12-1.pdf}

In a Bayesian statistics, a linear model will tell us the posterior
probability of its parameters (i.e., the effect size of a predictor)
given the data. Ignoring the marginal likelihood (or ``probability of
data''), this is what that looks like:

\[
P(\text{model parameters}|\text{data}) \propto P(\text{data}|\text{model parameters})*P(\text{model parameters})
\]

It's useful to formally define our model. We start with the likelihood,
or the probability of a particular parameter value given the observed
data (i.e., the first hald of the right-hand side of the equation). It
is a little tricky to think about the most appropriate probability
distribution for Bao's CV. The data-generating process---local
adaptation to different niches---might lead to gamma-distributed body
sizes (in this case, right-skewed data). Gamma distributed data seem
tricky to deal with in linear models. By log-transforming Bao's CV for
intertegular distance we should be able to use a Gaussian distribution,
which will make calculating phylgoenetic signal easier. The likehood
thus tells us Bao's CV is stochastically related to a normal
distribution with parameters \(\mu_i\) (the mean) and \(\sigma\)
(variance). In this case, \(\mu_i\) is itself \emph{deterministically}
related to the parameter \(\beta\) (``number of sites''), with an
overall intercept (\(\alpha\)) plus a set of phylogenetically correlated
intercepts (\(\alpha_j\)):

\[
\text{log(Bao's CV)} \sim \text{Normal}(\mu_i, \sigma)
\] \[
\mu_i = \alpha + \alpha_j + \beta_{\text{num_sites},i}
\]

We then need to define our priors, or the distribution of sensible
values for model parameters. We know that \(\alpha_j\) should be drawn
from a normal distribution with mean \(\alpha\) and correlation
structure \(\sigma_A\):

\[
\alpha_j \sim \text{Normal}(\alpha, \sigma_A)
\]

At this point it's probably easiest to get the rest of the priors we
need from the \texttt{get\_prior} function from \texttt{brms}:

\begin{Shaded}
\begin{Highlighting}[]
\NormalTok{priors }\OtherTok{\textless{}{-}} \FunctionTok{get\_prior}\NormalTok{(bao\_cv }\SpecialCharTok{\textasciitilde{}} \DecValTok{1} \SpecialCharTok{+}\NormalTok{ num\_sites }\SpecialCharTok{+}\NormalTok{ (}\DecValTok{1}\SpecialCharTok{|}\FunctionTok{gr}\NormalTok{(GENUS\_SPECIES, }\AttributeTok{cov =}\NormalTok{ A)),}
                    \AttributeTok{data =}\NormalTok{ bao\_values, }\AttributeTok{data2 =} \FunctionTok{list}\NormalTok{(}\AttributeTok{A =}\NormalTok{ A), }\AttributeTok{family=}\FunctionTok{gaussian}\NormalTok{())   }
\NormalTok{prior}
\end{Highlighting}
\end{Shaded}

\begin{verbatim}
## function (prior, ...) 
## {
##     call <- as.list(match.call()[-1])
##     seval <- rmNULL(call[prior_seval_args()])
##     call[prior_seval_args()] <- NULL
##     call <- lapply(call, deparse_no_string)
##     do_call(set_prior, c(call, seval))
## }
## <bytecode: 0x1431793d0>
## <environment: namespace:brms>
\end{verbatim}

This gives us a flat prior for the effect of num\_sites on the outcome
variable, e.g.~a uniform distribution over all reals:

\[
\beta_{\text{num_sites},i} \sim \text{Uniform}(-\infty,\infty)
\]

It also suggests that we use a standard Student's T distribution for
\(\sigma\), the standard deviation of the residuals:

\[
\sigma \sim \text{Student}(3, 0, 2.5)
\]

Now, we'll run our model, using an MCMC chain with the default number of
steps:

\begin{verbatim}
## Running /Library/Frameworks/R.framework/Resources/bin/R CMD SHLIB foo.c
## using C compiler: ‘Apple clang version 15.0.0 (clang-1500.0.40.1)’
## using SDK: ‘’
## clang -arch arm64 -I"/Library/Frameworks/R.framework/Resources/include" -DNDEBUG   -I"/Library/Frameworks/R.framework/Versions/4.4-arm64/Resources/library/Rcpp/include/"  -I"/Library/Frameworks/R.framework/Versions/4.4-arm64/Resources/library/RcppEigen/include/"  -I"/Library/Frameworks/R.framework/Versions/4.4-arm64/Resources/library/RcppEigen/include/unsupported"  -I"/Library/Frameworks/R.framework/Versions/4.4-arm64/Resources/library/BH/include" -I"/Library/Frameworks/R.framework/Versions/4.4-arm64/Resources/library/StanHeaders/include/src/"  -I"/Library/Frameworks/R.framework/Versions/4.4-arm64/Resources/library/StanHeaders/include/"  -I"/Library/Frameworks/R.framework/Versions/4.4-arm64/Resources/library/RcppParallel/include/"  -I"/Library/Frameworks/R.framework/Versions/4.4-arm64/Resources/library/rstan/include" -DEIGEN_NO_DEBUG  -DBOOST_DISABLE_ASSERTS  -DBOOST_PENDING_INTEGER_LOG2_HPP  -DSTAN_THREADS  -DUSE_STANC3 -DSTRICT_R_HEADERS  -DBOOST_PHOENIX_NO_VARIADIC_EXPRESSION  -D_HAS_AUTO_PTR_ETC=0  -include '/Library/Frameworks/R.framework/Versions/4.4-arm64/Resources/library/StanHeaders/include/stan/math/prim/fun/Eigen.hpp'  -D_REENTRANT -DRCPP_PARALLEL_USE_TBB=1   -I/opt/R/arm64/include    -fPIC  -falign-functions=64 -Wall -g -O2  -c foo.c -o foo.o
## In file included from <built-in>:1:
## In file included from /Library/Frameworks/R.framework/Versions/4.4-arm64/Resources/library/StanHeaders/include/stan/math/prim/fun/Eigen.hpp:22:
## In file included from /Library/Frameworks/R.framework/Versions/4.4-arm64/Resources/library/RcppEigen/include/Eigen/Dense:1:
## In file included from /Library/Frameworks/R.framework/Versions/4.4-arm64/Resources/library/RcppEigen/include/Eigen/Core:19:
## /Library/Frameworks/R.framework/Versions/4.4-arm64/Resources/library/RcppEigen/include/Eigen/src/Core/util/Macros.h:679:10: fatal error: 'cmath' file not found
## #include <cmath>
##          ^~~~~~~
## 1 error generated.
## make: *** [foo.o] Error 1
## 
## SAMPLING FOR MODEL 'anon_model' NOW (CHAIN 1).
## Chain 1: 
## Chain 1: Gradient evaluation took 5e-05 seconds
## Chain 1: 1000 transitions using 10 leapfrog steps per transition would take 0.5 seconds.
## Chain 1: Adjust your expectations accordingly!
## Chain 1: 
## Chain 1: 
## Chain 1: Iteration:    1 / 2000 [  0%]  (Warmup)
## Chain 1: Iteration:  200 / 2000 [ 10%]  (Warmup)
## Chain 1: Iteration:  400 / 2000 [ 20%]  (Warmup)
## Chain 1: Iteration:  600 / 2000 [ 30%]  (Warmup)
## Chain 1: Iteration:  800 / 2000 [ 40%]  (Warmup)
## Chain 1: Iteration: 1000 / 2000 [ 50%]  (Warmup)
## Chain 1: Iteration: 1001 / 2000 [ 50%]  (Sampling)
## Chain 1: Iteration: 1200 / 2000 [ 60%]  (Sampling)
## Chain 1: Iteration: 1400 / 2000 [ 70%]  (Sampling)
## Chain 1: Iteration: 1600 / 2000 [ 80%]  (Sampling)
## Chain 1: Iteration: 1800 / 2000 [ 90%]  (Sampling)
## Chain 1: Iteration: 2000 / 2000 [100%]  (Sampling)
## Chain 1: 
## Chain 1:  Elapsed Time: 0.287 seconds (Warm-up)
## Chain 1:                0.169 seconds (Sampling)
## Chain 1:                0.456 seconds (Total)
## Chain 1: 
## 
## SAMPLING FOR MODEL 'anon_model' NOW (CHAIN 2).
## Chain 2: 
## Chain 2: Gradient evaluation took 7e-06 seconds
## Chain 2: 1000 transitions using 10 leapfrog steps per transition would take 0.07 seconds.
## Chain 2: Adjust your expectations accordingly!
## Chain 2: 
## Chain 2: 
## Chain 2: Iteration:    1 / 2000 [  0%]  (Warmup)
## Chain 2: Iteration:  200 / 2000 [ 10%]  (Warmup)
## Chain 2: Iteration:  400 / 2000 [ 20%]  (Warmup)
## Chain 2: Iteration:  600 / 2000 [ 30%]  (Warmup)
## Chain 2: Iteration:  800 / 2000 [ 40%]  (Warmup)
## Chain 2: Iteration: 1000 / 2000 [ 50%]  (Warmup)
## Chain 2: Iteration: 1001 / 2000 [ 50%]  (Sampling)
## Chain 2: Iteration: 1200 / 2000 [ 60%]  (Sampling)
## Chain 2: Iteration: 1400 / 2000 [ 70%]  (Sampling)
## Chain 2: Iteration: 1600 / 2000 [ 80%]  (Sampling)
## Chain 2: Iteration: 1800 / 2000 [ 90%]  (Sampling)
## Chain 2: Iteration: 2000 / 2000 [100%]  (Sampling)
## Chain 2: 
## Chain 2:  Elapsed Time: 0.277 seconds (Warm-up)
## Chain 2:                0.134 seconds (Sampling)
## Chain 2:                0.411 seconds (Total)
## Chain 2: 
## 
## SAMPLING FOR MODEL 'anon_model' NOW (CHAIN 3).
## Chain 3: 
## Chain 3: Gradient evaluation took 8e-06 seconds
## Chain 3: 1000 transitions using 10 leapfrog steps per transition would take 0.08 seconds.
## Chain 3: Adjust your expectations accordingly!
## Chain 3: 
## Chain 3: 
## Chain 3: Iteration:    1 / 2000 [  0%]  (Warmup)
## Chain 3: Iteration:  200 / 2000 [ 10%]  (Warmup)
## Chain 3: Iteration:  400 / 2000 [ 20%]  (Warmup)
## Chain 3: Iteration:  600 / 2000 [ 30%]  (Warmup)
## Chain 3: Iteration:  800 / 2000 [ 40%]  (Warmup)
## Chain 3: Iteration: 1000 / 2000 [ 50%]  (Warmup)
## Chain 3: Iteration: 1001 / 2000 [ 50%]  (Sampling)
## Chain 3: Iteration: 1200 / 2000 [ 60%]  (Sampling)
## Chain 3: Iteration: 1400 / 2000 [ 70%]  (Sampling)
## Chain 3: Iteration: 1600 / 2000 [ 80%]  (Sampling)
## Chain 3: Iteration: 1800 / 2000 [ 90%]  (Sampling)
## Chain 3: Iteration: 2000 / 2000 [100%]  (Sampling)
## Chain 3: 
## Chain 3:  Elapsed Time: 0.262 seconds (Warm-up)
## Chain 3:                0.172 seconds (Sampling)
## Chain 3:                0.434 seconds (Total)
## Chain 3: 
## 
## SAMPLING FOR MODEL 'anon_model' NOW (CHAIN 4).
## Chain 4: 
## Chain 4: Gradient evaluation took 8e-06 seconds
## Chain 4: 1000 transitions using 10 leapfrog steps per transition would take 0.08 seconds.
## Chain 4: Adjust your expectations accordingly!
## Chain 4: 
## Chain 4: 
## Chain 4: Iteration:    1 / 2000 [  0%]  (Warmup)
## Chain 4: Iteration:  200 / 2000 [ 10%]  (Warmup)
## Chain 4: Iteration:  400 / 2000 [ 20%]  (Warmup)
## Chain 4: Iteration:  600 / 2000 [ 30%]  (Warmup)
## Chain 4: Iteration:  800 / 2000 [ 40%]  (Warmup)
## Chain 4: Iteration: 1000 / 2000 [ 50%]  (Warmup)
## Chain 4: Iteration: 1001 / 2000 [ 50%]  (Sampling)
## Chain 4: Iteration: 1200 / 2000 [ 60%]  (Sampling)
## Chain 4: Iteration: 1400 / 2000 [ 70%]  (Sampling)
## Chain 4: Iteration: 1600 / 2000 [ 80%]  (Sampling)
## Chain 4: Iteration: 1800 / 2000 [ 90%]  (Sampling)
## Chain 4: Iteration: 2000 / 2000 [100%]  (Sampling)
## Chain 4: 
## Chain 4:  Elapsed Time: 0.262 seconds (Warm-up)
## Chain 4:                0.113 seconds (Sampling)
## Chain 4:                0.375 seconds (Total)
## Chain 4:
\end{verbatim}

(Output curtailed to keep this document a manageable size/)

Let's summarize the results:

\begin{Shaded}
\begin{Highlighting}[]
\FunctionTok{summary}\NormalTok{(model\_1)}
\end{Highlighting}
\end{Shaded}

\begin{verbatim}
##  Family: gaussian 
##   Links: mu = identity; sigma = identity 
## Formula: log(bao_cv) ~ num_sites + (1 | gr(GENUS_SPECIES, cov = A)) 
##    Data: bao_values (Number of observations: 59) 
##   Draws: 4 chains, each with iter = 2000; warmup = 1000; thin = 1;
##          total post-warmup draws = 4000
## 
## Multilevel Hyperparameters:
## ~GENUS_SPECIES (Number of levels: 59) 
##               Estimate Est.Error l-95% CI u-95% CI Rhat Bulk_ESS Tail_ESS
## sd(Intercept)     0.34      0.10     0.15     0.53 1.01      531      966
## 
## Regression Coefficients:
##           Estimate Est.Error l-95% CI u-95% CI Rhat Bulk_ESS Tail_ESS
## Intercept    -3.24      0.36    -3.95    -2.52 1.00     2833     2070
## num_sites     0.02      0.02    -0.02     0.07 1.00     3213     2201
## 
## Further Distributional Parameters:
##       Estimate Est.Error l-95% CI u-95% CI Rhat Bulk_ESS Tail_ESS
## sigma     0.25      0.05     0.16     0.35 1.01      588      993
## 
## Draws were sampled using sampling(NUTS). For each parameter, Bulk_ESS
## and Tail_ESS are effective sample size measures, and Rhat is the potential
## scale reduction factor on split chains (at convergence, Rhat = 1).
\end{verbatim}

This suggests there is a positive effect of abundance on Bao's CV of
intertegular distance---but that it is not credible at the 95\% level.
It also suggests that phylogenetically correlated intercepts explain a
lot of the variation in Bao's CV we see here. We can formally calculate
Pagel's \(\lambda\)---the proportion of variance in the model
attributable to phylogeny---using its residuals. We can also perform a
one-sided hypothesis test to evaluate the evidence that \(\lambda > 0\).
To do so, we calculate the ratio of the posterior probabilty that
\(\lambda>0\) compared with the posterior probability that
\(\lambda=0\):

\begin{Shaded}
\begin{Highlighting}[]
\NormalTok{hyp }\OtherTok{\textless{}{-}} \StringTok{"sd\_GENUS\_SPECIES\_\_Intercept\^{}2 / (sd\_GENUS\_SPECIES\_\_Intercept\^{}2 + sigma\^{}2) = 0"}
\NormalTok{(hyp }\OtherTok{\textless{}{-}} \FunctionTok{hypothesis}\NormalTok{(model\_1, hyp, }\AttributeTok{class =} \ConstantTok{NULL}\NormalTok{))}
\end{Highlighting}
\end{Shaded}

\begin{verbatim}
## Hypothesis Tests for class :
##                 Hypothesis Estimate Est.Error CI.Lower CI.Upper Evid.Ratio
## 1 (sd_GENUS_SPECIES... = 0     0.62      0.19     0.18     0.89         NA
##   Post.Prob Star
## 1        NA    *
## ---
## 'CI': 90%-CI for one-sided and 95%-CI for two-sided hypotheses.
## '*': For one-sided hypotheses, the posterior probability exceeds 95%;
## for two-sided hypotheses, the value tested against lies outside the 95%-CI.
## Posterior probabilities of point hypotheses assume equal prior probabilities.
\end{verbatim}

\begin{Shaded}
\begin{Highlighting}[]
\FunctionTok{plot}\NormalTok{(hyp)}
\end{Highlighting}
\end{Shaded}

\includegraphics{01_analysis_files/figure-latex/unnamed-chunk-16-1.pdf}

This appears to be concordant with Laura's estimates from
\texttt{phytools}. The obvious next steps would be to see whether signal
declines using small subsets of the data (e.g., ask whether \(\lambda\)
only emerges at broader spatial scales). But we should chat first, I
think!

\end{document}
